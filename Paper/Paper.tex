\documentclass[11pt,preprint, authoryear]{elsarticle}

\usepackage{lmodern}
%%%% My spacing
\usepackage{setspace}
\setstretch{1.2}
\DeclareMathSizes{12}{14}{10}{10}

% Wrap around which gives all figures included the [H] command, or places it "here". This can be tedious to code in Rmarkdown.
\usepackage{float}
\let\origfigure\figure
\let\endorigfigure\endfigure
\renewenvironment{figure}[1][2] {
    \expandafter\origfigure\expandafter[H]
} {
    \endorigfigure
}

\let\origtable\table
\let\endorigtable\endtable
\renewenvironment{table}[1][2] {
    \expandafter\origtable\expandafter[H]
} {
    \endorigtable
}


\usepackage{ifxetex,ifluatex}
\usepackage{fixltx2e} % provides \textsubscript
\ifnum 0\ifxetex 1\fi\ifluatex 1\fi=0 % if pdftex
  \usepackage[T1]{fontenc}
  \usepackage[utf8]{inputenc}
\else % if luatex or xelatex
  \ifxetex
    \usepackage{mathspec}
    \usepackage{xltxtra,xunicode}
  \else
    \usepackage{fontspec}
  \fi
  \defaultfontfeatures{Mapping=tex-text,Scale=MatchLowercase}
  \newcommand{\euro}{€}
\fi

\usepackage{amssymb, amsmath, amsthm, amsfonts}

\def\bibsection{\section*{References}} %%% Make "References" appear before bibliography


\usepackage[round]{natbib}

\usepackage{longtable}
\usepackage[margin=2.3cm,bottom=2cm,top=2.5cm, includefoot]{geometry}
\usepackage{fancyhdr}
\usepackage[bottom, hang, flushmargin]{footmisc}
\usepackage{graphicx}
\numberwithin{equation}{section}
\numberwithin{figure}{section}
\numberwithin{table}{section}
\setlength{\parindent}{0cm}
\setlength{\parskip}{1.3ex plus 0.5ex minus 0.3ex}
\usepackage{textcomp}
\renewcommand{\headrulewidth}{0.2pt}
\renewcommand{\footrulewidth}{0.3pt}

\usepackage{array}
\newcolumntype{x}[1]{>{\centering\arraybackslash\hspace{0pt}}p{#1}}

%%%%  Remove the "preprint submitted to" part. Don't worry about this either, it just looks better without it:
\makeatletter
\def\ps@pprintTitle{%
  \let\@oddhead\@empty
  \let\@evenhead\@empty
  \let\@oddfoot\@empty
  \let\@evenfoot\@oddfoot
}
\makeatother

 \def\tightlist{} % This allows for subbullets!

\usepackage{hyperref}
\hypersetup{breaklinks=true,
            bookmarks=true,
            colorlinks=true,
            citecolor=blue,
            urlcolor=blue,
            linkcolor=blue,
            pdfborder={0 0 0}}


% The following packages allow huxtable to work:
\usepackage{siunitx}
\usepackage{multirow}
\usepackage{hhline}
\usepackage{calc}
\usepackage{tabularx}
\usepackage{booktabs}
\usepackage{caption}


\newenvironment{columns}[1][]{}{}

\newenvironment{column}[1]{\begin{minipage}{#1}\ignorespaces}{%
\end{minipage}
\ifhmode\unskip\fi
\aftergroup\useignorespacesandallpars}

\def\useignorespacesandallpars#1\ignorespaces\fi{%
#1\fi\ignorespacesandallpars}

\makeatletter
\def\ignorespacesandallpars{%
  \@ifnextchar\par
    {\expandafter\ignorespacesandallpars\@gobble}%
    {}%
}
\makeatother

\newlength{\cslhangindent}
\setlength{\cslhangindent}{1.5em}
\newenvironment{CSLReferences}%
  {\setlength{\parindent}{0pt}%
  \everypar{\setlength{\hangindent}{\cslhangindent}}\ignorespaces}%
  {\par}


\urlstyle{same}  % don't use monospace font for urls
\setlength{\parindent}{0pt}
\setlength{\parskip}{6pt plus 2pt minus 1pt}
\setlength{\emergencystretch}{3em}  % prevent overfull lines
\setcounter{secnumdepth}{5}

%%% Use protect on footnotes to avoid problems with footnotes in titles
\let\rmarkdownfootnote\footnote%
\def\footnote{\protect\rmarkdownfootnote}
\IfFileExists{upquote.sty}{\usepackage{upquote}}{}

%%% Include extra packages specified by user
\usepackage{booktabs}
\usepackage{longtable}
\usepackage{array}
\usepackage{multirow}
\usepackage{wrapfig}
\usepackage{float}
\usepackage{colortbl}
\usepackage{pdflscape}
\usepackage{tabu}
\usepackage{threeparttable}
\usepackage{threeparttablex}
\usepackage[normalem]{ulem}
\usepackage{makecell}
\usepackage{xcolor}

%%% Hard setting column skips for reports - this ensures greater consistency and control over the length settings in the document.
%% page layout
%% paragraphs
\setlength{\baselineskip}{12pt plus 0pt minus 0pt}
\setlength{\parskip}{12pt plus 0pt minus 0pt}
\setlength{\parindent}{0pt plus 0pt minus 0pt}
%% floats
\setlength{\floatsep}{12pt plus 0 pt minus 0pt}
\setlength{\textfloatsep}{20pt plus 0pt minus 0pt}
\setlength{\intextsep}{14pt plus 0pt minus 0pt}
\setlength{\dbltextfloatsep}{20pt plus 0pt minus 0pt}
\setlength{\dblfloatsep}{14pt plus 0pt minus 0pt}
%% maths
\setlength{\abovedisplayskip}{12pt plus 0pt minus 0pt}
\setlength{\belowdisplayskip}{12pt plus 0pt minus 0pt}
%% lists
\setlength{\topsep}{10pt plus 0pt minus 0pt}
\setlength{\partopsep}{3pt plus 0pt minus 0pt}
\setlength{\itemsep}{5pt plus 0pt minus 0pt}
\setlength{\labelsep}{8mm plus 0mm minus 0mm}
\setlength{\parsep}{\the\parskip}
\setlength{\listparindent}{\the\parindent}
%% verbatim
\setlength{\fboxsep}{5pt plus 0pt minus 0pt}



\begin{document}



\begin{frontmatter}  %

\title{The Impact of Loadshedding on SWIX Sectors using DCC-GARCH
models}

% Set to FALSE if wanting to remove title (for submission)




\author[Add1]{Sven Wellmann}
\ead{20850980@sun.ac.za}





\address[Add1]{Stellenbosch University, Stellenbosch, South Africa}


\begin{abstract}
\small{
Abstract
}
\end{abstract}

\vspace{1cm}





\vspace{0.5cm}

\end{frontmatter}



%________________________
% Header and Footers
%%%%%%%%%%%%%%%%%%%%%%%%%%%%%%%%%
\pagestyle{fancy}
\chead{}
\rhead{}
\lfoot{}
\rfoot{\footnotesize Page \thepage}
\lhead{}
%\rfoot{\footnotesize Page \thepage } % "e.g. Page 2"
\cfoot{}

%\setlength\headheight{30pt}
%%%%%%%%%%%%%%%%%%%%%%%%%%%%%%%%%
%________________________

\headsep 35pt % So that header does not go over title




\hypertarget{introduction}{%
\section{\texorpdfstring{Introduction
\label{Introduction}}{Introduction }}\label{introduction}}

\hypertarget{literature-review}{%
\section{Literature Review}\label{literature-review}}

\hypertarget{data}{%
\section{\texorpdfstring{Data \label{Data}}{Data }}\label{data}}

The aim of this project is to construct time-varying conditional
correlations between different sectors of the ALSI. These co-movements
between sectors are then analysed over a stratified period where South
Africa was affected by load-shedding. This study utilises a series of
daily returns and weights of stocks in the ALSI, which are then filtered
into either the Industrial, Financial, Resource or Property sector.
Sector returns are calculated on a daily level and then the entire data
set is subset into months than South Africa experienced load-shedding
and months where it did not. The sample period of the data is from 1
January 2014 until 31 October 2022.

The sector returns are plotted in Figure \ref{CumRet} where it is
visible that the sectors show characteristics of co-movement, especially
during market shocks. This is evident with the onset of COVID-19 in
March of 2020 where all sectors demonstrate a significant drawback in
cumulative returns. Since this downturn in 2020, the Resource sector has
grown significantly while Property has not been able to recover to
pre-pandemic levels.

\begin{figure}[H]

{\centering \includegraphics{Paper_files/figure-latex/CumRet-1} 

}

\caption{Cumulative Returns per Sector for ALSI and SWIX \label{CumRet}}\label{fig:CumRet}
\end{figure}

\hypertarget{summary-statistics}{%
\subsection{Summary Statistics}\label{summary-statistics}}

Summary statistics for the series of daily sector returns are shown in
Table \ref{table1}. Resources show the largest average daily return
(\(\mu = 0.0005\)) for the sample period, while Property shows the
lowest average daily return (\(\mu = 0\)). The largest standard
deviation belongs to the Resource sector, indicating that it has the
largest volatility from a historical and static perspective. The
Property sector has both the largest daily drawdown and the largest
daily recovery over the sample period, at -19.35\% and 15.49\%. All of
the sectors show negatively skewed tails, indicating that large negative
returns are more likely than large positive returns for this sample
period. Property exhibits a high degree of kurtosis, having its
distribution concentrated around the mean. The Jarque-Bera test is used
to assess whether the daily returns follow a normal distribution. The
significance stars in Table \ref{table1} below indicate that the null
hypothesis of normality in the sampled returns is rejected for all
sectors at a 1\% significance level.

\begin{table}

\caption{\label{tab:table1}Summary Statistics and Test Scores for Sectors \label{table1}}
\centering
\fontsize{9}{11}\selectfont
\begin{tabular}[t]{l|l|l|l|l}
\hline
  & Financials & Industrials & Property & Resources\\
\hline
Mean & 0.0004 & 0.0003 & 0 & 0.0005\\
\hline
Median & 0.0006 & 0.0006 & -0.0002 & 0.0004\\
\hline
Std.Dev & 0.0161 & 0.0126 & 0.0171 & 0.0185\\
\hline
Min & -0.1227 & -0.0881 & -0.1935 & -0.1457\\
\hline
Max & 0.0885 & 0.0761 & 0.1549 & 0.1353\\
\hline
Skewness & -0.3902 & -0.1333 & -0.837 & -0.1218\\
\hline
Kurtosis & 6.3353 & 3.5137 & 26.4552 & 5.0679\\
\hline
Observations & 2208 & 2208 & 2208 & 2208\\
\hline
Jarque.Bera & 3758.53* & 1146.21* & 64776.75* & 2375.19*\\
\hline
Ljung.Box & 2100.8* & 652.49* & 1839.24* & 1171.94\\
\hline
LM.GARCH & 1331.899* & 347.2372* & 930.0183* & 649.1169*\\
\hline
\multicolumn{5}{l}{\textsuperscript{} Note: This table provides summary statistics for daily returns of}\\
\multicolumn{5}{l}{the sectors used in our study. Sample period: 1 January 2014 to 31}\\
\multicolumn{5}{l}{October 2022. * denotes statistical significance at 1\%}\\
\end{tabular}
\end{table}

\hypertarget{serial-autocorrelation-and-arch-effects}{%
\subsection{Serial Autocorrelation and ARCH
Effects}\label{serial-autocorrelation-and-arch-effects}}

The Ljung-Box test is used to examine the data for serial
autocorrelation over longer time periods, therefore a lag length of 10
is used. The results in Table \ref{table1} show that the null hypothesis
of no autocorrelation can be rejected, requiring autoregressive terms in
the mean equations to be fitted on all of the data. To supplement the
Ljung-Box test, Figure \ref{LogRet} plots the daily returns. Figure
\ref{LogRet} displays periods of volatility clustering, otherwise known
as market momentum. This is a strong indication of second order
persistence in the time series, pointing to autoregressive conditional
heteroskedasticity (ARCH) effects and long memory that require explicit
modelling of the variance components.

Engle (\protect\hyperlink{ref-engle}{1982}) LM-GARCH test confirms the
presence of ARCH effects in all the series using:
\[\epsilon_t^2= \beta_0 + (\Sigma_{s = 1}^{10}\beta_s\epsilon_{t-s}^2) + v_t \].
To control for these ARCH effects, Engle
(\protect\hyperlink{ref-engle}{1982}) showed that it is possible to
simultaneously model the mean and variance equations of a series using
GARCH models. This technique will be further utilized to extract the
time-varying conditional correlations of our time series data.

\begin{figure}[H]

{\centering \includegraphics{Paper_files/figure-latex/LogRet-1} 

}

\caption{Log Returns per Sector for the SWIX \label{LogRet}}\label{fig:LogRet}
\end{figure}

\hypertarget{co-integration-tests}{%
\subsection{Co-Integration Tests}\label{co-integration-tests}}

The next step is to test for co-integration in order to motivate the
study of the time series conditional correlations. The Johansen
(\protect\hyperlink{ref-johansen1988}{1988}) co-integration test will be
used to confirm whether there is at least one linear long-run
relationship among the time series that would yield stationary
residuals. The Johansen co-integration test uses a Vector Error
Correction Model (VECM) approach with the form:

\begin{align}
  \Delta p_t = \prod p_{t-k} + \Gamma_1 p_{t-1} + \Gamma_2 p_{t-2} + ... + \Gamma_{7} p_{t-7} + \mu + \delta (t) + \theta D_t + \epsilon_t \label{eq1}
\end{align}

where \(p_t\) is a \((1 \times 4)\) vector of the daily returns for the
sectors at time \(t\). The Johansen test centres around the examination
of the \(\prod\)-matrix. The \(\prod\)-matrix has form
\(\prod = \alpha \beta^{'}\) where \(\beta\) is the \(k^{th}\) order
co-integrating vector and \(\alpha\) is the adjustment parameter. Below
the Trace and Maximum Eigenvalue tests are used to consider the rank of
the \(\prod\)-matrix using its eigenvalues. The rank will give an
indication of long-run dependence. The Trace statistic tests whether the
number of co-integrating vectors of the system is less than or equal to
4, while the Max-Eigenvalue statistic reflects separate tests used on
each eigenvalue of the \(\prod\)-matrix. If the tests indicate that the
rank of \(\prod\) is statistically likely to be greater than 0, it would
imply that there is co-integration and a long-run relationship between
the variables.

The findings from the Trace test is that all relationships are
co-integrating at the 1\% significance level. The Eigenvalue test is
conducted and confirms this result. This could indicate that the sectors
have a common underlying factor that affects their movements. The
co-integrating relationship represents correlation between the time
series processes in the long term and validates the necessity to study
these co-movements.

\hypertarget{unconditional-coerrelation}{%
\subsection{Unconditional
Coerrelation}\label{unconditional-coerrelation}}

Table \ref{corrtab} gives the unconditional correlation of returns of
the sectors for the entire period as well as for the unconditional
correlation for the period where South Africa experienced load-shedding.

These static estimates of historic correlation are often used in
practice but have many limitations. The unconditional correlations are
limited by the sample period and do not account for any changes.
Therefore they do not accurately reflect the relationships between the
time series at specific points in time. In this section, it has been
observed that the time series data of the sector returns display both
first and second order serial autocorrelation. Thus the static estimates
of correlation are misleading when the mean persistence and conditional
heteroskedasticity is not controlled for. Static estimates of
correlation also fail to take into account the dynamic nature of the
underlying correlations. In the next section the underlying correlations
conditional on past information will be studied.

\begin{table}

\caption{\label{tab:corrtab}Static Unconditional Correlation \label{corrtab}}
\centering
\fontsize{9}{11}\selectfont
\begin{tabular}[t]{l|r|r|r|r}
\hline
  & Financials & Industrials & Property & Resources\\
\hline
Financials & 1.0000 & 0.5779 & 0.6855 & 0.5005\\
\hline
Industrials & 0.5096 & 1.0000 & 0.5267 & 0.5171\\
\hline
Property & 0.6212 & 0.3557 & 1.0000 & 0.4156\\
\hline
Resources & 0.3770 & 0.4679 & 0.2746 & 1.0000\\
\hline
\multicolumn{5}{l}{\textsuperscript{} Note: This table provides correlation for daily returns of}\\
\multicolumn{5}{l}{the different sectors over the whole period (bottom left)}\\
\multicolumn{5}{l}{and just the periods of load-shedding (top right).}\\
\end{tabular}
\end{table}

\hypertarget{methodology}{%
\section{\texorpdfstring{Methodology
\label{Meth}}{Methodology }}\label{methodology}}

\hypertarget{dcc-model}{%
\subsection{DCC Model}\label{dcc-model}}

The goal of this study is to examine the dynamic correlations between
the Industrial, Financial, Property and Resource sectors of the ALSI as
well as to understand how these correlations change over time. To
achieve this goal, the DCC (Dynamic Conditional Correlation) model is
used. The DCC model is a statistical model that is commonly used to
analyse the correlations between multiple time series data. It consists
of two components: a GARCH (Generalized Autoregressive Conditional
Heteroscedasticity) model, which is used to model the variance of the
time series data, and a dynamic conditional correlation model, which is
used to model the correlations between the time series.

To estimate the DCC model the maximum likelihood estimation procedure is
used. This involves specifying the functional form of the model,
including the functional form of the GARCH and dynamic conditional
correlation components. Then estimating the parameters of the model that
maximize the likelihood of the data given the model. The DCC model has
been widely used in the literature to study dynamic correlations in
financial markets.

This study will use the sector daily returns, a \(1 \times 4\)
stochastic vector \(\{r_t\}\). The DCC model is specified as follows:

\begin{align}
  r_{it} = \mu_{it} + \epsilon_{it} \label{eq2}
\end{align}

\begin{align}
  \epsilon_{it} = \sqrt{H_{it}}. \eta_i  \text{  with  } \epsilon_{it} \sim N(0, H_t)  \text{  and  } \eta_i \sim N(0, I) . \label{eq3}
\end{align}

Where \(\mu_t\) is the unconditional AR(1)-mean equation, \(\epsilon_t\)
the vector of ordinary residuals, \(H_t\) the \(N \times N\) conditional
covariance matrix and \(\eta_i\) the standardised residuals.

Various MV-GARCH models have been proposed to model the covariance
process, \(H_t\), in Equation \ref{eq3}. The DCC model will be used in
this paper which allows the covariance matrix to be separated into
different univariate volatility equations and their respective
conditional correlations.

The covariance matrix in the DCC-Model is defined as follows:

\begin{align}
  H_t = D_tR_tD_t \label{eq4}
\end{align}

with \(D_t = diag(\sqrt{h_{11,t}}...\sqrt{h_{NN,t}})\) and \(h_{ii,t}\)
taking the functional form of the univariate GARCH model specified. The
dynamic conditional correlation structure is then given by the following
equation:

\begin{align}
  Q_{ij,t} = (1 - \theta_1 - \theta_2).\bar{Q} + \theta_1(\epsilon_{i,t-1}\epsilon_{j,t-1}^{'}) + \theta_2(Q_{ij,t-1})  \label{eq5}
\end{align}

where \(Q_{ij,t}\) is the unconditional variance between different
series \(i\) and \(j\), \(\bar{Q}\) is the unconditional covariance
between the series estimated in the univariate GARCH specification and
estimation step. The scalar parameters \(\theta_1\) and \(\theta_2\)
must satisfy both the non negativity assumption \(\theta_1 \geq 0\) and
\(\theta_2 \geq 0\) as well as the assumption that
\(\theta_1 + \theta_2 < 1\). The second step the requires only the
estimate of \(\theta_1\) and \(\theta_2\) using a likelihood function.
Equation \ref{5} expresses the unconditional variance matrix,
\(Q_{ij,t}\), as a standard GARCH-type equation, so that we can derive
the dynamic conditional correlation matrix, \(R_t\), between any two
series as:

\begin{align}
  R_t = Q_{ij,t}^{*-1}. Q_{ij,t}. Q_{ij,t}^{*-1} \label{eq6}
\end{align}

with \(Q_{ij,t}^{*}\) a diagonal matrix with the square root of the
diagonal elements of \(Q_{ij,t}\) as its entries, such that
\(Q_{ij,t}^{*} = Diag(Q_t)^{1/2}\). This process can be thought of
intuitively as multiplying both sides of Equation \ref{6} by the inverse
of Diagonal matrix \(D_t\). The dynamic conditional correlation matrix,
\(R_{ij,t}\) will therefore have entries in the bivariate framework as
follows:

\begin{align}
  \rho_{ij,t} = \dfrac{q_{ij,t}}{\sqrt{q_{ii,t}.q_{jj,t}}} \\ 
  = \dfrac{(1 - \theta_1 - \theta_2) \bar{q} + \theta_1 \epsilon_{i,t-1} \epsilon_{j,t-1}^{'} + \theta_2 q_{ij,t-1}}{((1 - \theta_1 - \theta_2)\bar{q_i} + \theta_1     \epsilon_{i,t-1}^2 + \theta_2 q_{ii,t-1}) . ((1 - \theta_1 - \theta_2) \bar{q_{j}} + \theta_1 \epsilon_{j,t-1}^2 + \theta_2 q_{jj,t-1})} \label{eq7}
\end{align}

Following the methodology of Engle
(\protect\hyperlink{ref-engle2002}{2002}), the DCC model is estimated by
maximising the log-likelihood function for Equation \ref{eq2} as:

\begin{align}
  L(\theta, \phi) = -\dfrac{1}{2} \sum_{t=1}^T (ln(2\pi) + ln(\left|D_tR_tD_t\right|) + \epsilon_t^{'}(D_tR_tD_t)^{-1}\epsilon_t)  \label{eq8}
\end{align}

and using the fact that \(H_t = D_tR_tD_t\), Equation \ref{eq8} can be
simplified as:

\begin{align}
  L(\theta, \phi) = -\dfrac{T}{2} ln(2\pi) - \dfrac{1}{2} . \sum_{t=1}^T (2 . ln \left| D_t \right| + \epsilon_t^{'} (D_tD_t)^{-1} \epsilon_t) - \dfrac{1}{2}\sum_{t=1}^T (ln\left| R_t \right| + \epsilon_t^{'} (R_t^{-1}) \epsilon_t)  \label{eq9}
\end{align}

The second step is then to maximise the correlation part by using the
maximised value in the equation above to solve:

\begin{align}
  L_C(\theta, \phi) = -\dfrac{1}{2}\sum_{t=1}^T (ln\left| R_t \right| + \epsilon_t^{'} (R_t^{-1}) \epsilon_t) \label{eq10}
\end{align}

The two-stage DCC estimation procedure outlined above has parameter
estimates that are both consistent and asymptotically normal. Cappiello,
Engle \& Sheppard
(\protect\hyperlink{ref-cappiello2006asymmetric}{2006}) find a clear
limitation of the DDC, that the dynamics of the conditional correlation
do not account for asymmetric effects. This means that although the
model accounts for the magnitude of past shocks' impact on future
conditional volatility and correlation, it does not differentiate
between positive and negative shock effects. This limitation will not be
resolved in this study.
\footnote{In order to account for these effects the Asymmetric Dynamic Conditional Correlation (ADCC) was created, for more information follow @cappiello2006asymmetric or @engle1995multivariate}

After the GARCH model is fit at a univariate level, the dynamic
conditional correlation series will be fit for each bi-variate
relationship. This dynamic structure of the returns of each sector pair
of Resources, Financials, Property and Industrials will be explored
further.

\hypertarget{univariate-specification}{%
\subsection{Univariate Specification}\label{univariate-specification}}

The first stage in building the DCC model framework consists of fitting
univariate GARCH specifications to each series of returns. The
univariate GARCH model fit on the data is the GJR-GARCH
{[}glosten1993relation{]}. Table \ref{table:univar} below contains the
chosen univariate GJR-GARCH fit.

The Univariate GARCH model includes:

Mean equation: \begin{align}
  r_t = \mu + a_1. r_{t-1} + \epsilon_t  \label{eq11}
\end{align}

Volatility equation:

\begin{align}
  \epsilon_t = \sqrt{h_t}.\eta_t, \quad \text{where} \eta_t \sim N(0,1)  \label{eq12}
\end{align}

where \(r_t\) is the daily return for a sector at time \(t\), \(\mu\) is
the unconditional mean of the time series and \(a_1\) is the coefficient
for the autoregressive term in the mean equation. The conditional
variance at time \(t\), \(h_t\), is modeled as a function of the
variance at previous time periods and the error term at previous time
periods. The random error term, \(\epsilon_t\), is multiplied by the
square root of the conditional variance to ensure that the variance is
always positive. The standardized error term, \(\eta_t\), has a mean of
0 and a standard deviation of 1. Equation \ref{eq12} specifies the
relationship between the error term \(\epsilon_t\) and the conditional
variance \(h_t\).

The GJR-GARCH model is a variant of the GARCH model that can effectively
capture fat tails, excess kurtosis, and leverage effects, allowing for
more flexibility in modeling the time-varying variance. In the
GJR-GARCH(1,1) model, the variance at time \(t\) is modelled as:

\begin{align}
  h_t^2 = \beta_0 + \beta_1( |\epsilon_{t-1}| - \gamma \epsilon_{t-1})^2 + \beta_2 h_{t-1}^2  \label{eq13}
\end{align}

This model combines elements of both the ARCH and GARCH models, and
different model specifications can be obtained by varying the parameters
\(\gamma\) and \(\beta_1\). For example, setting \(\gamma = 0\) results
in a GARCH model.

The parameter \(\gamma\) in GJR-GARCH model reflects the leverage
effect, which refers to the phenomenon where the magnitude of the
response of the variance to a shock depends on the sign of the shock. A
positive value of \(\gamma\) indicates that negative shocks tend to have
a larger impact on the variance than positive shocks of the same
magnitude, while a negative value of \(\gamma\) indicates the opposite.
This model specifications allow the GJR-GARCH model to capture different
responses of the variance to positive and negative shocks, which can be
important for accurately modelling time series data that exhibits
asymmetry.

Table \ref{univar} shows that the returns series for all sectors display
strong persistence in volatility and both ARCH and GARCH effects, as
measured by \((\alpha_1 + \beta_2 )\). This indicates volatility
clustering. The statistical significance of the \(\beta_2\) parameter
for all sectors in the GJF-GARCH model indicates a strong presence of
conditional heteroskedasticity. This is consistent with the results of
Engle (\protect\hyperlink{ref-engle}{1982}) LM-GARCH test run in section
\ref{Data} and confirms the presence of ARCH effects. This further
suggests that the static measures of return correlations between sector
returns in the last section are not reliable.

The main takeaway from Table \ref{univar} comes from the \(\alpha_1\)
and \(\beta_2\) parameters, which can be used to assess the persistence
of volatility in the time series. \(\alpha_1\) measures the contribution
of past shocks to the current variance, while \(\beta_1\) measures the
contribution of the past variance to the current variance. A positive
value of \(\alpha_1\) indicates that positive shocks tend to be followed
by more volatility than negative shocks of a similar magnitude, while a
negative value of \(\alpha_1\) indicates the opposite. Table
\ref{univar} shows that Financials and Industrials have a negative value
for \(\alpha_1\) indicating an asymmetry where negative shocks have a
larger impact on volatility than positive shocks. Property and Resources
both have positive values for \(\alpha_1\). Similarly, a positive value
of \(\beta_2\) indicates that high levels of volatility tend to persist
over time, while a negative value of \(\beta_2\) indicates that high
levels of volatility tend to dissipate over time. All sectors have a
significant and positive \(\beta_2\). All sectors also have a positive
and significant \(\gamma\) indicating that negative shocks tend to have
a larger impact on the variance than positive shocks of the same
magnitude.

To estimate the time-varying DCC model, the next step is to extract the
standardized residuals from the estimated GJR GARCH model and maximize
the log-likelihood function. This allows for the estimation of the
time-varying conditional correlations between the sectors of the ALSI.

\begin{table}[H]

\caption{\label{tab:univar}Univariate GJR GARCH Coefficients \label{univar}}
\centering
\resizebox{\linewidth}{!}{
\begin{tabular}[t]{l|r|r|r|r|r|r|r|r}
\hline
Sector & $a_0$ & $a_1$ & $\beta_0$ & $\beta_1$ & $\beta_2$ & $\gamma$ & Skewness & Shape\\
\hline
Financials & 0.0003 & -0.0044 & 0.0000 & 0.0241 & 0.9176 & 0.0906 & 0.9546 & 8.0904\\
\hline
 & 0.2075 & 0.8405 & 0.2271 & 0.0980 & 0.0000 & 0.0001 & 0.0000 & 0.0000\\
\hline
Industrials & 0.0002 & -0.0084 & 0.0000 & 0.0057 & 0.9017 & 0.1295 & 0.9117 & 8.8658\\
\hline
 & 0.2538 & 0.6971 & 0.0000 & 0.1845 & 0.0000 & 0.0000 & 0.0000 & 0.0000\\
\hline
Property & -0.0001 & 0.0005 & 0.0000 & 0.0374 & 0.8854 & 0.1012 & 1.0024 & 5.4217\\
\hline
 & 0.6317 & 0.9809 & 0.0527 & 0.0000 & 0.0000 & 0.0000 & 0.0000 & 0.0000\\
\hline
Resources & 0.0003 & 0.0066 & 0.0000 & 0.0169 & 0.9444 & 0.0626 & 0.9554 & 10.3411\\
\hline
 & 0.3851 & 0.7588 & 0.4273 & 0.2334 & 0.0000 & 0.0001 & 0.0000 & 0.0000\\
\hline
\multicolumn{9}{l}{\textsuperscript{} Note: This table provides the univariate GJR GARCH model coefficients with the p-values underneath.}\\
\end{tabular}}
\end{table}

\hypertarget{results}{%
\section{Results}\label{results}}

In order to understand the dynamics of the different sectors of the ALSI
the DCC models are analysed. The DCC model separates the variance and
the correlation structure into two distinct components: the GARCH
component, which models the variance of each time series, and the
dynamic conditional correlation component, which models the correlations
between the time series. This allows the DCC model to capture both the
direct and indirect effects of one time series on the variance of
another time series. Additionally, the DCC model allows for the dynamic
correlations between time series to vary over time.

\hypertarget{whole-period}{%
\subsection{Whole Period}\label{whole-period}}

\begin{table}

\caption{\label{tab:dccfitw}..... \label{dccfitw}}
\centering
\fontsize{9}{11}\selectfont
\begin{tabular}[t]{l|r|r|r|r}
\hline
  &  Estimate &  Std. Error &  t value & Pr(>|t|)\\
\hline
[Financials].omega & 0.0000 & 0.0000 & 0.6434 & 0.5199\\
\hline
[Financials].alpha1 & 0.0241 & 0.0248 & 0.9719 & 0.3311\\
\hline
[Financials].beta1 & 0.9176 & 0.0471 & 19.4922 & 0.0000\\
\hline
[Industrials].omega & 0.0000 & 0.0000 & 5.3197 & 0.0000\\
\hline
[Industrials].alpha1 & 0.0057 & 0.0071 & 0.8065 & 0.4199\\
\hline
[Industrials].beta1 & 0.9017 & 0.0092 & 98.1795 & 0.0000\\
\hline
[Property].omega & 0.0000 & 0.0000 & 0.6647 & 0.5062\\
\hline
[Property].alpha1 & 0.0374 & 0.0397 & 0.9403 & 0.3471\\
\hline
[Property].beta1 & 0.8854 & 0.0178 & 49.6809 & 0.0000\\
\hline
[Resources].omega & 0.0000 & 0.0000 & 0.3505 & 0.7260\\
\hline
[Resources].alpha1 & 0.0169 & 0.0301 & 0.5623 & 0.5739\\
\hline
[Resources].beta1 & 0.9444 & 0.0479 & 19.7292 & 0.0000\\
\hline
[Joint]dcca1 & 0.0266 & 0.0047 & 5.6699 & 0.0000\\
\hline
[Joint]dccb1 & 0.9515 & 0.0090 & 105.1736 & 0.0000\\
\hline
\multicolumn{5}{l}{\textsuperscript{} Note: This table displays the fit of the dynamic conditional}\\
\multicolumn{5}{l}{correlations of daily returns of different sectors within the}\\
\multicolumn{5}{l}{ALSI over the full period of this study.}\\
\end{tabular}
\end{table}

\begin{figure}[H]

{\centering \includegraphics{Paper_files/figure-latex/DCCfullr-1} 

}

\caption{Dynamic Conditional Correlations: Resources \label{DCCfullr}}\label{fig:DCCfullr}
\end{figure}

\begin{figure}[H]

{\centering \includegraphics{Paper_files/figure-latex/DCCfullf-1} 

}

\caption{Dynamic Conditional Correlations: Financials \label{DCCfullf}}\label{fig:DCCfullf}
\end{figure}

\begin{figure}[H]

{\centering \includegraphics{Paper_files/figure-latex/DCCfulli-1} 

}

\caption{Dynamic Conditional Correlations: Industrials \label{DCCfulli}}\label{fig:DCCfulli}
\end{figure}

\begin{figure}[H]

{\centering \includegraphics{Paper_files/figure-latex/DCCfullp-1} 

}

\caption{Dynamic Conditional Correlations: Property \label{DCCfullp}}\label{fig:DCCfullp}
\end{figure}

\begin{table}
\centering\begingroup\fontsize{9}{11}\selectfont

\begin{tabular}{l|r|r|r|r}
\hline
  & Mean & Std.Dev & Min & Max\\
\hline
Financials -> Industrials & 0.4856 & 0.1342 & -0.0124 & 0.7865\\
\hline
Financials -> Resources & 0.5443 & 0.0949 & 0.2640 & 0.8294\\
\hline
Financials -> Property & 0.2869 & 0.1331 & -0.2362 & 0.6137\\
\hline
Industrials -> Resources & 0.3089 & 0.1252 & -0.0669 & 0.7016\\
\hline
Industrials -> Property & 0.4283 & 0.0969 & 0.1348 & 0.7545\\
\hline
Resources -> Property & 0.1646 & 0.1257 & -0.1447 & 0.5596\\
\hline
\multicolumn{5}{l}{\textsuperscript{} Note: This table provides summary statistics of the dynamic}\\
\multicolumn{5}{l}{conditional correlations of daily returns of different sectors within}\\
\multicolumn{5}{l}{the ALSI over the entire period of this study.}\\
\end{tabular}
\endgroup{}
\end{table}

\hypertarget{load-shedding-period}{%
\subsection{Load Shedding Period}\label{load-shedding-period}}

\begin{table}
\centering\begingroup\fontsize{9}{11}\selectfont

\begin{tabular}{l|r|r|r|r}
\hline
  &  Estimate &  Std. Error &  t value & Pr(>|t|)\\
\hline
[Financials].omega & 0.0000 & 0.0000 & 2.6464 & 0.0081\\
\hline
[Financials].alpha1 & 0.0000 & 0.0758 & 0.0000 & 1.0000\\
\hline
[Financials].beta1 & 0.7239 & 0.0818 & 8.8533 & 0.0000\\
\hline
[Industrials].omega & 0.0000 & 0.0000 & 18.2457 & 0.0000\\
\hline
[Industrials].alpha1 & 0.0000 & 0.0143 & 0.0000 & 1.0000\\
\hline
[Industrials].beta1 & 0.8280 & 0.0183 & 45.1670 & 0.0000\\
\hline
[Property].omega & 0.0001 & 0.0000 & 3.1694 & 0.0015\\
\hline
[Property].alpha1 & 0.0687 & 0.0954 & 0.7200 & 0.4716\\
\hline
[Property].beta1 & 0.5314 & 0.1099 & 4.8351 & 0.0000\\
\hline
[Resources].omega & 0.0000 & 0.0000 & 2.5240 & 0.0116\\
\hline
[Resources].alpha1 & 0.0000 & 0.0279 & 0.0000 & 1.0000\\
\hline
[Resources].beta1 & 0.7978 & 0.0585 & 13.6449 & 0.0000\\
\hline
[Joint]dcca1 & 0.0247 & 0.0102 & 2.4348 & 0.0149\\
\hline
[Joint]dccb1 & 0.9275 & 0.0314 & 29.5384 & 0.0000\\
\hline
\multicolumn{5}{l}{\textsuperscript{} Note: This table displays the fit of the dynamic conditional}\\
\multicolumn{5}{l}{correlations of daily returns of different sectors within the}\\
\multicolumn{5}{l}{ALSI over the periods that encounter loadshedding in this study.}\\
\end{tabular}
\endgroup{}
\end{table}

\begin{figure}[H]

{\centering \includegraphics{Paper_files/figure-latex/DCClsr-1} 

}

\caption{Dynamic Conditional Correlations: Resources  \label{DCClsr}}\label{fig:DCClsr}
\end{figure}

\begin{figure}[H]

{\centering \includegraphics{Paper_files/figure-latex/DCClsi-1} 

}

\caption{Dynamic Conditional Correlations: Industrials g \label{DCClsi}}\label{fig:DCClsi}
\end{figure}

\begin{figure}[H]

{\centering \includegraphics{Paper_files/figure-latex/DCClsf-1} 

}

\caption{Dynamic Conditional Correlations: Financials \label{DCClsf}}\label{fig:DCClsf}
\end{figure}

\begin{figure}[H]

{\centering \includegraphics{Paper_files/figure-latex/DCClsp-1} 

}

\caption{Dynamic Conditional Correlations: Property  \label{DCClsp}}\label{fig:DCClsp}
\end{figure}

\begin{table}
\centering\begingroup\fontsize{9}{11}\selectfont

\begin{tabular}{l|r|r|r|r}
\hline
  & Mean & Std.Dev & Min & Max\\
\hline
Financials -> Industrials & 0.5077 & 0.0763 & 0.2413 & 0.7059\\
\hline
Financials -> Resources & 0.5344 & 0.0655 & 0.3827 & 0.7042\\
\hline
Financials -> Property & 0.3516 & 0.0914 & 0.1279 & 0.5758\\
\hline
Industrials -> Resources & 0.4565 & 0.0730 & 0.3156 & 0.7085\\
\hline
Industrials -> Property & 0.4050 & 0.1006 & 0.0987 & 0.6634\\
\hline
Resources -> Property & 0.2957 & 0.0824 & 0.0803 & 0.5228\\
\hline
\multicolumn{5}{l}{\textsuperscript{} Note: This table provides summary statistics of the dynamic}\\
\multicolumn{5}{l}{conditional correlations of daily returns of different sectors}\\
\multicolumn{5}{l}{within the ALSI over the periods that encounter loadshedding in this}\\
\multicolumn{5}{l}{study.}\\
\end{tabular}
\endgroup{}
\end{table}

The results from the Dynamic Conditional Correlations (DCC) model show
that the volatility of sector returns exhibit strong co-movements, as
observed in the Table \ref{table:descdcc} and illustrated in the Figure
\ref{fig:dcc}. The DCC model, which extracts the underlying process and
removes noise from the data, reveals lower correlations compared to the
static correlations presented in section \ref{unconcorr}. The minimum
and maximum values of the correlations tend to occur during times of
endogenous and exogenous shocks to the market. This suggests that the
market is highly interconnected and exhibits volatility clustering, as
found in section \ref{volclust}.

When examining the dynamic correlations between sectors in Figure
\ref{fig:btcdcc}, it is clear that the co-movements between these assets
are relatively low.

\hfill

\hypertarget{conclusion}{%
\section{Conclusion}\label{conclusion}}

\newpage

\hypertarget{references}{%
\section*{References}\label{references}}
\addcontentsline{toc}{section}{References}

\hypertarget{refs}{}
\begin{CSLReferences}{1}{0}
\leavevmode\vadjust pre{\hypertarget{ref-cappiello2006asymmetric}{}}%
Cappiello, L., Engle, R.F. \& Sheppard, K. 2006. Asymmetric dynamics in
the correlations of global equity and bond returns. \emph{Journal of
Financial econometrics}. 4(4):537--572.

\leavevmode\vadjust pre{\hypertarget{ref-engle2002}{}}%
Engle, R. 2002. Dynamic conditional correlation: A simple class of
multivariate generalized autoregressive conditional heteroskedasticity
models. \emph{Journal of Business \& Economic Statistics}.
20(3):339--350.

\leavevmode\vadjust pre{\hypertarget{ref-engle}{}}%
Engle, R.F. 1982. Autoregressive conditional heteroscedasticity with
estimates of the variance of united kingdom inflation.
\emph{Econometrica: Journal of the econometric society}. 987--1007.

\leavevmode\vadjust pre{\hypertarget{ref-johansen1988}{}}%
Johansen, S. 1988. Statistical analysis of cointegration vectors.
\emph{Journal of economic dynamics and control}. 12(2-3):231--254.

\end{CSLReferences}

\hypertarget{appendix}{%
\section*{Appendix}\label{appendix}}
\addcontentsline{toc}{section}{Appendix}

\hypertarget{no-loadshedding-results}{%
\subsection{No Loadshedding Results}\label{no-loadshedding-results}}

\begin{table}
\centering\begingroup\fontsize{9}{11}\selectfont

\begin{tabular}{l|r|r|r|r}
\hline
  &  Estimate &  Std. Error &  t value & Pr(>|t|)\\
\hline
[Financials].omega & 0.0000 & 0.0000 & 0.5307 & 0.5956\\
\hline
[Financials].alpha1 & 0.0308 & 0.0202 & 1.5247 & 0.1273\\
\hline
[Financials].beta1 & 0.9319 & 0.0296 & 31.4997 & 0.0000\\
\hline
[Industrials].omega & 0.0000 & 0.0000 & 0.1620 & 0.8713\\
\hline
[Industrials].alpha1 & 0.0203 & 0.0968 & 0.2099 & 0.8337\\
\hline
[Industrials].beta1 & 0.9021 & 0.0823 & 10.9579 & 0.0000\\
\hline
[Property].omega & 0.0000 & 0.0000 & 0.4572 & 0.6475\\
\hline
[Property].alpha1 & 0.0594 & 0.0502 & 1.1813 & 0.2375\\
\hline
[Property].beta1 & 0.8878 & 0.0279 & 31.7715 & 0.0000\\
\hline
[Resources].omega & 0.0000 & 0.0000 & 0.4453 & 0.6561\\
\hline
[Resources].alpha1 & 0.0393 & 0.0206 & 1.9053 & 0.0567\\
\hline
[Resources].beta1 & 0.9418 & 0.0283 & 33.2911 & 0.0000\\
\hline
[Joint]dcca1 & 0.0279 & 0.0061 & 4.5714 & 0.0000\\
\hline
[Joint]dccb1 & 0.9440 & 0.0143 & 65.8566 & 0.0000\\
\hline
\multicolumn{5}{l}{\textsuperscript{} Note: This table displays the fit of the dynamic conditional}\\
\multicolumn{5}{l}{correlations of daily returns of different sectors within the}\\
\multicolumn{5}{l}{ALSI over the periods that do not encounter loadshedding in this}\\
\multicolumn{5}{l}{study.}\\
\end{tabular}
\endgroup{}
\end{table}

\begin{table}
\centering\begingroup\fontsize{9}{11}\selectfont

\begin{tabular}{l|r|r|r|r}
\hline
  & Mean & Std.Dev & Min & Max\\
\hline
Financials -> Industrials & 0.4809 & 0.1287 & 0.0128 & 0.7520\\
\hline
Financials -> Resources & 0.5440 & 0.0877 & 0.2343 & 0.8391\\
\hline
Financials -> Property & 0.2744 & 0.1260 & -0.2475 & 0.5506\\
\hline
Industrials -> Resources & 0.2642 & 0.1138 & -0.1141 & 0.5330\\
\hline
Industrials -> Property & 0.4384 & 0.0794 & 0.1904 & 0.6814\\
\hline
Resources -> Property & 0.1434 & 0.1158 & -0.1522 & 0.5349\\
\hline
\multicolumn{5}{l}{\textsuperscript{} Note: This table provides summary statistics of the dynamic}\\
\multicolumn{5}{l}{conditional correlations of daily returns of different sectors}\\
\multicolumn{5}{l}{within the ALSI over the periods that do not encounter loadshedding}\\
\multicolumn{5}{l}{in this study.}\\
\end{tabular}
\endgroup{}
\end{table}

\bibliography{Tex/ref}





\end{document}
